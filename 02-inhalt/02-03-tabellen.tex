\section{Tabellen}
\label{sec:tabellen}

Anstatt \texttt{multicolumn} kann auch einfach nur \texttt{mc} benutzt werden.


\ref{tbl:eigenschaftsuebersicht-der-drohne} zeigt die Systemparameter eines Flugroboters der sitebots GmbH.
\begin{table}[h]
  \centering
  \begin{threeparttable}
    \makebox[\linewidth]{%
      \renewcommand{\TPTminimum}{\linewidth}
      \begin{tabular}{crr@{ }l}
        \toprule

        Eigenschaft & \mc{3}{c}{Wert} \\

        \cmidrule[0.4pt](r{0.25em}){1-1} \cmidrule[0.4pt](lr{0.25em}){2-4}

        Gewicht (ohne Beladung) & $=$ & 4 & \SI{}{\kilogram} \\
        Gewicht (mit Beladung) & $=$ & 5,2 & \SI{}{\kilogram} \\
        durchschnittliche Flugdauer\tnote{1)} & $\approx$ & 20 & \SI{}{\minute} \\
        Durchmesser & $=$ & 1 & \SI{}{\meter} \\
        Fluggeschwindigkeit\tnote{2)} & $\leq$ & 20 & \SI{}{\meter\per\second} \\
        Flughöhe\tnote{3)} & $\leq$ & 100 & \SI{}{\meter} \\

        \bottomrule
      \end{tabular}
    }
    \begin{tablenotes}
      \footnotesize
      \item[1)] Entspricht der durchschnittlichen Flugdauer pro Akkuladung und kann durch Wechseln von Akkus beliebig erweitert werden.
      \item[2)] Bei einer Geschwindigkeit von \SI{20}{\meter\per\second} beträgt der Anstellwinkel \SI{20}{\degree} über der Horizontalen.
      \item[3)] In der Praxis auf \SI{100}{\meter} begrenzt durch die allgemeine Aufstiegserlaubnis über Grund.
    \end{tablenotes}
  \end{threeparttable}
  \caption[Systemparameter der Drohne]{Systemparameter der Drohne nach \citet[S.~9]{Bachelorthesis.2016}}
  \label{tbl:eigenschaftsuebersicht-der-drohne}
\end{table}


\begin{table}[ht]
  \centering
  % Tabular example from LaTeX manual, p.205
  \begin{tabular}{|r||r@{--}l|p{38mm}|}
    \hline
    \mc{4}{|c|}{GG\&A Hoofed Stock}\\ \hline \hline
     & \mc{2}{c|}{Price} & \\ \cline{2-3}
    \mc{1}{|c||}{Year} & \mc{1}{r@{\,\vline\,}}{low} & high & \mc{1}{c|}{Comments} \\ \hline
    1971 & 97 & 245 & Bad year for farmers in the west. \\ \hline
      72 & 245 & 245 & Light trading due to a heavy winter. \\ \hline
      73 & 245 & 2001 & No gnus was very good gnus this year. \\ \hline
  \end{tabular}
  \caption[GG\&A Hoofed Stock]{GG\&A Hoofed Stock. Tabelle mit Priotität der Position unmittelbar an der Stelle im Quellcode durch kennzeichnen mit \texttt{ht}. Abgeändert übernommen von \url{https://mirror.hmc.edu/ctan//info/examples/lb2/5-7-8.ltx}.}
\end{table}
