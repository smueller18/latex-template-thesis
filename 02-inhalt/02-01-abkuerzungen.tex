\section{Abkürzungen}
\label{sec:Abkürzungen}

\subsection{\acrlongpl{bps}}

\gu Das menschliche Gehirn nimmt ab etwa 14 bis 16 \acrlongpldat{bps} (individuell verschieden) aufeinanderfolgende Bilder als bewegte (aber nicht unbedingt ruckelfreie) Szene wahr, weswegen die Bildfrequenz in der Anfangszeit der bewegten Bilder (Stummfilmzeit), nach einer experimentellen Phase, auf 16 \acrlong{bps} festgelegt wurde. Auf dem zweiten internationalen Kongress der Filmhersteller von Paris 1909 wurden 1000 Bilder in der Minute festgelegt, was in etwa 16–17 \acrlongpl{bps} entspricht. Viele späte Stummfilme wurden jedoch mit höheren Bildfrequenzen, wie z. B. 22 \acrlongpldat{bps}, aufgenommen. Mit der Einführung des Tonfilms wurde die Bildfrequenz auf 24 Hz festgelegt.\go \citep{www:wikipedia:bildfrequenz.2017}.
