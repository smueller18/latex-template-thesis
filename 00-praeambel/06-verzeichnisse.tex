% Inhaltsverzeichnis
\setcounter{secnumdepth}{2}
\renewcommand*{\theparagraph}{\thesubsubsection.\alph{paragraph}}
\setcounter{tocdepth}{2}
\usepackage{tocstyle}
\usetocstyle{allwithdot}

% Anhang richtig verlinkt im Inhaltsverzeichnis
\renewcommand*{\theHsection}{\thesection}


% Tabellen- und Abbildungsverzeichnis
% Einrücken verhindern
% Default: 1.5em/2.3em
\makeatletter
  \renewcommand*\l@figure{\@dottedtocline{1}{0em}{2.3em}}
  \let\l@table\l@figure
\makeatother

\renewcommand{\thefigure}{Abb.~\arabic{chapter}.\arabic{figure}}
\renewcommand{\thetable}{Tabelle~\arabic{chapter}.\arabic{table}}
\renewcommand{\theequation}{Gl.~\arabic{chapter}.\arabic{equation}}
\renewcommand*{\figureformat}{\thefigure}
\renewcommand*{\tableformat}{\thetable}
\renewcommand*{\captionformat}{: }

% Bild-/Tabellenunterschriften klein & serifenlos
\addtokomafont{caption}{\small\sffamily}
% Bild-/Tabelllabel fett & serifenlos
\addtokomafont{captionlabel}{\sffamily\bfseries}

% Doppelpunkt nach Nummern
\AtBeginDocument{
  \addtocontents{lof}{\protect\def\protect\autodot{:}}
  \addtocontents{lot}{\protect\def\protect\autodot{:}}
}

\usepackage{caption}

% Abkürzungsverzeichnis
\usepackage[
  % fügt das Verzeichnis dem Inhaltsverzeichnis zu
	toc,
  % Abkürzungsverzeichnis hinzufügen
  acronym,
  % keine Seitenzahlen
  nonumberlist,
  % kein Punkt am Ende der Beschreibung
  nopostdot,
  % Kein Abstand zwischen Alphabetgruppen
  nogroupskip
]{glossaries}

\newglossary[slg]{symbols}{sls}{slo}{Symbolverzeichnis}

% Benutzerdefinierte Acronymeigenschaften
% user1
\let\acrlonggen\glsuseri
% user2
\let\acrlongpldat\glsuserii

% Style auf long aufbauend aber mit verbessertem Abstand
\newglossarystyle{longspace}{%
  \setglossarystyle{long}
  \renewenvironment{theglossary}%
     {\begin{longtable}[l]{@{}p{0.15\textwidth}@{}p{0.85\textwidth}}}
     {\end{longtable}}
  \renewcommand{\glossentry}[2]{%
    \glsentryitem{##1}\glstarget{##1}{\glossentryname{##1}} &
    \glossentrydesc{##1}\glspostdescription\space ##2\tabularnewline[5mm]
  }%
  \renewcommand{\subglossentry}[3]{%
     &
     \glssubentryitem{##2}%
     \glstarget{##2}{\strut}\glossentrydesc{##2}\glspostdescription\space
     ##3\tabularnewline[2mm]
  }%
}

% Generiere später die Glossary Dateien
\makeglossaries
