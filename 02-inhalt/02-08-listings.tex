\section{Listings}
\label{sec:listings}

Listing \ref{lst:punktinpolygon} ist direkt in das Latexdokument eingebettet.
\begin{code}
  \begin{lstlisting}[firstnumber=11,language=java]
public boolean enthaeltKoordinate( double lat, double lng ) {

    boolean ungAnzSchnittpkt = false;
    int i, j = polyKanten - 1;

    for( i = 0; i < polyKanten; i++ ) {
        if(( LngArr[ i ] < lng && LngArr[ j ] >= lng ) ||
           ( LngArr[ j ] < lng && LngArr[ i ] >= lng )) {
            if( LatArr[ i ] + (( lng - LngArr[ i ] ) /
              ( LngArr[ j ] - LngArr[ i ] ) * ( LatArr[ j ] - LatArr[ i ] )) >= lat ) {
                ungAnzSchnittpkt = !ungAnzSchnittpkt;
            }
        }
        j = i;
    }
    return ungAnzSchnittpkt;
} // Ende enthaeltKoordinate()-Methode
  \end{lstlisting}
  \caption[Auszug der PunktInPolygon-Klasse]{Auszug der \texttt{PunktInPolygon}-Klasse. Dieser zeigt die Implementierung des Punkt-in-Polygon Algorithmus.}
  \label{lst:punktinpolygon}
\end{code}


Der Quelltext aus Listing \ref{lst:collector-chillii} wird aus einer Datei geladen. Es wurde eine Funktion in die Einstellungen eingebaut, mit der das teilweise Anzeigen einer Quelltextdatei leicht möglich ist. Der Befehl lautet\\
\texttt{\scriptsize |\textbackslash nextline[Tabeinrückung]\{Zeilennummer der nächsten Zeile\}\{Kommentar für die ausgelassenen Zeilen\}|}.\\
Beispiel sind
\begin{enumerate} \footnotesize
  \item \texttt{|\textbackslash nextline\{24\}|}
  \item \texttt{|\textbackslash nextline[4]\{24\}|}
  \item \texttt{|\textbackslash nextline\{24\}\{Exception Handling\}|}
  \item \texttt{|\textbackslash nextline[4]\{24\}\{Exception Handling\}|}
\end{enumerate}




\begin{code}
  \lstinputlisting[language=Python]{sourcecode/collector.py}
  \caption[Collector-Skript für den System Controller]{Auszug aus dem Collector-Skript \texttt{chillii-collector/collector.py} für das Auslesen von Sensorwerten des System Controllers über die Modbus-Schnittstelle. Auszüge entnommen von \url{https://github.com/smueller18/chillii-collector/blob/1.0/collector.py}.}
  \label{lst:collector-chillii}
\end{code}
