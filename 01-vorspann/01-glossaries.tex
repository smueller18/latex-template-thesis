% Anleitung zu Glossaries und Acronyms:
% https://en.wikibooks.org/wiki/LaTeX/Glossary

\newacronym[
  plural={Bilder pro Sekunde},
  longplural={Bilder pro Sekunde},
  user2={Bildern pro Sekunde}
]{bps}{BpS}{Bild pro Sekunde}


\newglossaryentry{naiive}{
  name=na\"{\i}ve,
  description={is a French loanword (adjective, form of naïf) indicating having or showing a lack of experience, understanding or sophistication},
  sort=naiive
}

\newglossaryentry{Linux}{
  name=Linux,
  description={is a generic term referring to the family of Unix-like computer operating systems that use the Linux kernel},
  plural=Linuces
}

\longnewglossaryentry{computer}{name=computer}{
  is a programmable machine that receives input, stores and manipulates data, and provides output in a useful format
}
\newglossaryentry{real number}{
  name={real number},
  description={include both rational numbers, such as $42$ and $\frac{-23}{129}$, and irrational numbers, such as $\pi$ and the square root of two; or, a real number can be given by an infinite decimal representation, such as $2.4871773339\ldots$ where the digits continue in some way; or, the real numbers may be thought of as points on an infinitely long number line},
  symbol={\ensuremath{\mathbb{R}}}
}


\newglossaryentry{euro}{
  name={\texteuro},
  symbol={-},
  description={Währung der Europäischen Union},
  type=symbols,
  sort=euro
}

\newglossaryentry{dollar}{
  name={\textdollar},
  symbol={-},
  description={Währung der Vereinigten Staaten von Amerika},
  type=symbols,
  sort=dollar
}

\newglossaryentry{cent}{
  name={\textcent},
  symbol={-},
  description={Centbeträge von Währungen},
  type=symbols,
  sort=cent
}

\newglossaryentry{chf}{
  name={CHF},
  symbol={-},
  description={Währung der Schweiz},
  type=symbols,
  sort=chf
}

\newglossaryentry{gu}{
  name={\gu Text},
  symbol={-},
  description={Gänsefüßchen unten mit \texttt{\textbackslash gu Text}},
  type=symbols
}

\newglossaryentry{go}{
  name={Text\go},
  symbol={-},
  description={Gänsefüßchen oben mit \texttt{Text\textbackslash go}},
  type=symbols
}
