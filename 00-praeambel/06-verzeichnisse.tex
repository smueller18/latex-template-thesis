% Inhaltsverzeichnis
\setcounter{secnumdepth}{3}
\renewcommand*{\theparagraph}{\thesubsubsection.\alph{paragraph}}
\setcounter{tocdepth}{3}
\usepackage{tocstyle}
\usetocstyle{allwithdot}

% Anhang richtig verlinkt im Inhaltsverzeichnis
\renewcommand*{\theHsection}{\thesection}


% Tabellen- und Abbildungsverzeichnis
% Einrücken verhindern
% Default: 1.5em/2.3em
\makeatletter
  \renewcommand*\l@figure{\@dottedtocline{1}{0em}{2.3em}}
  \let\l@table\l@figure
\makeatother

\renewcommand{\thefigure}{Abb.~\arabic{chapter}.\arabic{figure}}
\renewcommand{\thetable}{Tabelle~\arabic{chapter}.\arabic{table}}
\renewcommand{\theequation}{Gl.~\arabic{chapter}.\arabic{equation}}
\renewcommand*{\figureformat}{\thefigure}
\renewcommand*{\tableformat}{\thetable}
\renewcommand*{\captionformat}{: }

% Bild-/Tabellenunterschriften klein & serifenlos
\addtokomafont{caption}{\small\sffamily}
% Bild-/Tabelllabel fett & serifenlos
\addtokomafont{captionlabel}{\sffamily\bfseries}

% Doppelpunkt nach Nummern
\AtBeginDocument{
  \addtocontents{lof}{\protect\def\protect\autodot{:}}
  \addtocontents{lot}{\protect\def\protect\autodot{:}}
}

\usepackage{caption}

% Abkürzungsverzeichnis
\usepackage[
  % fügt das Verzeichnis dem Inhaltsverzeichnis zu
	toc,
  % Abkürzungsverzeichnis hinzufügen
  acronym,
  % keine Seitenzahlen
  nonumberlist,
  % kein Punkt am Ende der Beschreibung
  nopostdot
]{glossaries}

\newglossary[slg]{symbols}{sls}{slo}{Symbolverzeichnis}

% Benutzerdefinierte Acronymeigenschaften
% user1
\let\acrlonggen\glsuseri
% user2
\let\acrlongpldat\glsuserii


% Erstelle Stye für das Symbolverzeichnis
\newglossarystyle{symbolstyle}{
  \renewenvironment{theglossary}{
    \begin{longtable}{@{}p{0.1\textwidth}@{}p{0.9\textwidth}}
  }{
    \end{longtable}
  }
  \renewcommand*{\glossaryheader}{}
  \renewcommand*{\glsgroupheading}[1]{}
  \renewcommand*{\glossaryentryfield}[5]{
    \glstarget{\textbf{##1}}{\sffett{\rlap{##2}}} & ##3\glspostdescription\space ##5\\
  }
  \renewcommand*{\glossarysubentryfield}[6]{
    \glossaryentryfield{##2}{##3}{##4}{##5}{##6}
  }
  \renewcommand*{\glsgroupskip}{ & \\
    \renewcommand{\arraystretch}{1.4}
  }
}


% Generiere später die Glossary Dateien
\makeglossaries
