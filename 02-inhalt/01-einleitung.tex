\chapter{Einleitung}
\label{chap:einleitung}


\section{Motivation und Problemstellung}
\label{sec:motivation-und-problemstellung}
Viele Studenten würden gerne auf die Nachteile von Office Word beim Schreiben einer Abschlussarbeit verzichten. Die Alternative dazu ist \LaTeX, ein qualitativ hochwertiges Schriftsatzprogramm. Leider ist die Bedienung für Einsteiger nicht gerade intuitiv und die anfängliche Einarbeitung in die neue Umgebung sehr mühsam. Mit einer pofessionallen Vorlage, die nach dem Herunterladen ohne weitere Anstrengungen sofort ein PDF Dokument erstellt wird, reduziert die anfängliche Frustration von Studenten in enormen Maße.

\section{Zielsetzung}
\label{sec:zielsetzung}
Mit dieser Arbeit soll Studenten eine Vorlage geboten werden, die alle wesentlichen Bestandteile einer Forschungsarbeit aufzeigt. Angefangen von der initialen Einbindung zahlreicher Pakete über die Bereitstellung einer vollständig aufgesetzten Dokumentenstruktur bis hin zu umfangreichen Fallbeispielen, ist alles enthalten, was für eine erfolgreiche Umsetzung einer Abschlussarbeit benötigt wird.


\section{Aufbau der Arbeit}
\label{sec:struktur}
Die Struktur der Thesis ist wie folgt:
\begin{description}
  \item[Kapitel 2] beinhaltet umfangreiche Praxisbeispiele zu vielen verschiedenen Anwendungsgebieten. So wird zum Beispiel das Einbinden von Bildern, das Erstellen von Tabellen oder das Einbinden von Programmcode vorgeführt.
  \item[Kapitel 3] wird nur mit Blindtext gefüllt.
\end{description}
