\section{Formeln}

Seien $a$ und $b$ die Katheten und $c$ die Hypothenuse, dann gilt $c^{2} = a^{2} + b^{2}$ (Pythagoreischer Lehrsatz).

Seien $a$ und $b$ die Katheten und $c$ die Hypothenuse, dann gilt
  \[ c^{2} = a^{2} + b^{2} \]
(Pythagoreischer Lehrsatz).

Seien $a$ und $b$ die Katheten und $c$ die Hypothenuse, dann gilt
  \begin{equation}\label{eq:pyth2}
    c^{2} = a^{2} + b^{2}
  \end{equation}
(Pythagoreischer Lehrsatz).

\begin{eqnarray}
  f(x) & = & \cos x \nonumber\\
  f'(x) & = & -\sin x \nonumber\\
  \int_{0}^{x} f(t) dt & = & \sin(x)
\end{eqnarray}

Summenzeichen mit Indizes rechts ($\sum_{i=1}^{\infty} \frac{x^n}{n!}$) und unten bzw. oben ($\sum\limits_{i=1}^{\infty} \frac{x^n}{n!}$).

\ref{eq:matrix-x} zeigt eine Definition einer Matrix.
\begin{equation} \label{eq:matrix-x}
  X =
  \begin{pmatrix}
    x_{11} & x_{12} & \cdots  & x_{1n} \\
    x_{21} & x_{22}  & \cdots & x_{2n} \\
    \vdots  & \vdots & \ddots & \vdots \\
    x_{m1} & x_{m2} & \hdots  & x_{mn}
  \end{pmatrix}
  , x_{ij} \in \mathbb{R}
\end{equation}

\begin{equation}
f(x) =
  \left\{
    \begin{array}{l@{\quad}l}
      0, &  x\leq 0\\
      x^2 & x>0
    \end{array}
  \right.
\end{equation}
